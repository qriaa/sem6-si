\documentclass[12pt, a4paper]{article}
\usepackage[polish]{babel}
\usepackage[T1]{fontenc}
\usepackage{csquotes}
\usepackage{amsmath}
\usepackage{array}
\usepackage{amsfonts}
\usepackage{hyperref}
\usepackage{biblatex}
\addbibresource{citations.bib}

\title{Sztuczna inteligencja i inżynieria wiedzy \\ Lista 2}
\author{Gabriel Urbaniak 260428}
\date{\today}

\newtheorem{definition}{Definicja}

\begin{document}
\maketitle

\section{Problem}

\subsection{Reversi}
Reversi to gra:
\begin{itemize}
    \item dwuosobowa,
    \item o sumie zerowej,
    \item niekooperacyjna,
    \item całkowicie deterministyczna,
    \item w której każdy gracz ma pełną informację o stanie gry,
    \item o skończonym czasie rozgrywki.
\end{itemize}

Zadaniem gracza w Reversi jest zwycięstwo przez stawianie pionków na planszy w taki sposób, by przejąć pionki przeciwnika -
na koniec gry należy posiadać najwięcej pionków, by wygrać.

\begin{definition}[Gracz]
    Gracz $\boldsymbol{G} \in \{B, C\}$ to gracz grający w grę Reversi.
    Może wykonać ruch po ruchu przeciwnika lub jako pierwszy, gdy $\boldsymbol{G} = B$
\end{definition}

\begin{definition}[Plansza]
    Plansza to macierz $\boldsymbol{P} = (p_{ij})_{1 \le i \le 8,\text{ } 1 \le j \le 8}$, gdzie $p$ to pole na tej planszy.
    Pole może mieć trzy różne stany: $ \boldsymbol{S} = \{0, B, C\}, p_{ij} \in \boldsymbol{S} $.
\end{definition}

\begin{definition}[Ruch]
    Gracz może się ruszyć przez postawienie swojego \\ pionka na dowolne z wolnych pól $p_{ij} = 0$
    zmieniając jego symbol na $B$ lub $C$ zależnie od symbolu gracza dopóki jest w stanie przejąć pionki przeciwnika.
    Jeśli nie jest w stanie przejąć ani jednym ruchem pionka przeciwnika, to ruch gracza jest pomijany.
\end{definition}

\begin{definition}[Przejęcie]
    Przy postawieniu pionka jednego z graczy $\boldsymbol{G}$ na pole $p_{ij}$ należy sprawdzić, czy sąsiadujące pola
    $p_{i \pm 1 j \pm 1}$ należą do przeciwnika. Jeśli tak, to wszystkie pionki przeciwnika w linii prostej w
    pionie, poziomie lub przekątnej macierzy przecinającą pole $p_{ij}$ w kierunku tego pionka przeciwnika zostają przejęte
    przez gracza $\boldsymbol{G}$, czyli zmienione na jego symbol, pod warunkiem, że linia ta kończy się również jego pionkiem.
\end{definition}

\begin{definition}[Koniec gry]
    Gra się kończy, gdy żaden z graczy $\boldsymbol{G}$ nie może wykonać ruchu.
    Zliczane są wszystkie pola planszy $\boldsymbol{P}$ według ich stanu $\boldsymbol{S}$.
    Jeśli liczba czarnych pionków i białych jest taka sama, żaden z graczy nie wygrywa.
    W przeciwnym przypadku wygrywa gracz, który posiada na planszy więcej pionków od drugiego.
\end{definition}

\begin{definition}[Stan gry]
    Stan gry to $\boldsymbol{R_{s}} = (\boldsymbol{P}, \boldsymbol{G})$, gdzie $\boldsymbol{P}$ to plansza, a $\boldsymbol{G}$ to gracz,
    który miałby wykonywać teraz ruch. 
\end{definition}

\begin{definition}[Stan początkowy]
    Początkowym stanem gry jest:
    $$\boldsymbol{R_{s}}^\prime = (\boldsymbol{P}^\prime, B)$$
    $$
        \boldsymbol{P}^\prime = 
        \begin{bmatrix}
            0 & 0 & 0 & 0 & 0 & 0 & 0 & 0 \\
            0 & 0 & 0 & 0 & 0 & 0 & 0 & 0 \\
            0 & 0 & 0 & 0 & 0 & 0 & 0 & 0 \\
            0 & 0 & 0 & B & C & 0 & 0 & 0 \\
            0 & 0 & 0 & C & B & 0 & 0 & 0 \\
            0 & 0 & 0 & 0 & 0 & 0 & 0 & 0 \\
            0 & 0 & 0 & 0 & 0 & 0 & 0 & 0 \\
            0 & 0 & 0 & 0 & 0 & 0 & 0 & 0 \\
        \end{bmatrix}
    $$
\end{definition}

\subsection{Metoda}

Problem możnaby było rozwiązać przez sprawdzenie każdego możliwego ruchu jaki istnieje, 
jednak wymagałoby to ogromnej mocy obliczeniowej. Z tego powodu najlepszy możliwy ruch w danej sytuacji 
będzie znajdywany przez częściowe rozwinięcie drzewa gry. Aby ocenić grę, która jeszcze się nie zakończyła
należy stworzyć \textbf{funkcję oceniającą} która mogłaby wartością numeryczną ocenić dobroć sytuacji dla algorytmu
przeszukującego \textbf{drzewo decyzyjne}.

\begin{definition}[Drzewo decyzyjne]
    Drzewo decyzyjne to $\boldsymbol{N} = (C, S, \boldsymbol{R_s}), \\
    C = \{\boldsymbol{N}_1, \boldsymbol{N}_2, ...,  \boldsymbol{N}_n\} \lor \emptyset,
    S \in \mathbb{R} $, gdzie N jest węzłem, a S wynikiem funkcji oceniającej dla danego węzła.
\end{definition}

\begin{definition}[Funkcja oceniająca (heurystyczna)]
    $$f = w_1 \cdot x_1 + w_2 \cdot x_2 + ... + w_n \cdot x_n$$
    $f$ to funkcja oceniająca, gdzie $x_n$ to pewna miara stanu planszy,
    a $w_n$ to waga odpowiedniej miary stanu planszy.
\end{definition}

Algorytmem, który będzie wykorzystywał owe dane będzie \textbf{Minimax} oraz jego modyfikacja -
\textbf{Alfa-beta cięcie}, które pozwoli na omijanie węzłów które już nie ma sensu przeszukiwać, by
oszczędzić trochę mocy obliczeniowej w celu poprawy np. głębokości drzewa bądź czasu wykonania algorytmu.

\section{Implementacja}
Zasady gry zaimplementował Dr inż. Piotr Syga\cite{reversiimpl}. \\
Implementacja pozwala na zmierzenie się ze sobą dwóch dowolnych graczy,
gdzie graczami mogą być Minimax, Alfa-beta cięcie bądź człowiek.
Wybranymi miarami stanu planszy zostały:
\begin{itemize}
    \item Liczba posiadanych pionków
    \item Elastyczność - liczba dostępnych ruchów
    \item Blokowanie - ujemna liczba dostępnych ruchów przeciwnika (kara za to, że ma ruchy)
    \item Liczba pionków przeciwnika sąsiadująca z pionkami algorytmu \\
    (potencjalne okazje do przejęcia)
    \item Liczba pionków przeciwnika sąsiadujących z pustymi polami. \\
    (potencjalne okazje do przejęcia)
\end{itemize}
W programie można stworzyć własne tabele wag (strategii),
jednak zostało umieszczone w nim kilka przykładowych:
\begin{center}
    \begin{tabular}{| m{5em} | m{4em} | m{5.5em} | m{5em} | m{4em} | m{3em} |}
        \hline
        nazwa & liczba pionków & elastyczność & blokowanie & sąsiedzi pionków przeciwnika & okazje \\
        \hline
        aggressive & 4.0 & 2.0 & 0.5 & 1.5 & 2.0 \\
        \hline
        controlling & 1.0 & 5.0 & 5.0 & 1.0 & 1.0 \\
        \hline
        balanced & 1.0 & 2.0 & 2.0 & 1.0 & 1.0 \\
        \hline
    \end{tabular}
\end{center}

Kod znajduje się na \href{https://github.com/qriaa/sem6-si/tree/main/lista2}{tej stronie}.

\section{Rezultaty}
Podstawowymi mierzalnymi parametrami do porównania
są czas wykonania oraz ilość odwiedzonych węzłów.

Przedstawione zostaną dane przykładowej partii między algorytmem Minimax
a jego zmodyfikowanej wersji z Alfa-beta cięciem.

Parametrami obu algorytmów są: \\
\texttt{
    Głębokość: 4
    Strategia: balanced
}
\begin{center}
    \begin{tabular}{ | m{4em} | m{4.5em} | m{3em} | m{4em} | m{4.5em} | m{3em} |}
        \hline
        Minimax & czas & węzły & Alfa-beta & czas & węzły \\
        \hline
        (2, 4) & 0.100894 & 317 & (4, 5) & 0.063711 & 198 \\
        \hline
        (5, 2) & 0.267528 & 847 & (2, 3) & 0.112144 & 344 \\
        \hline
        (2, 5) & 0.82 & 2573 & (2, 2) & 0.246547 & 742 \\
        \hline
        (4, 2) & 1.18808 & 3736 & (5, 3) & 0.215184 & 654 \\
        \hline
        (3, 2) & 2.10616 & 6834 & (4, 1) & 0.635999 & 1949 \\
        \hline
        (3, 1) & 4.601079 & 15131 & (1, 5) & 0.680569 & 2200 \\
        \hline
        (5, 4) & 5.337107 & 18047 & (3, 0) & 0.761715 & 2505 \\
        \hline
        (1, 3) & 4.264432 & 14603 & (5, 5) & 0.673579 & 2231 \\
        \hline
        (3, 5) & 6.348387 & 22341 & (6, 5) & 0.903315 & 3040 \\
        \hline
        (6, 4) & 9.784564 & 35112 & (1, 2) & 0.829852 & 2851 \\
        \hline
        (5, 1) & 11.259568 & 40868 & (7, 4) & 1.229304 & 4242 \\
        \hline
        (5, 6) & 7.426985 & 27506 & (6, 6) & 2.254521 & 8138 \\
        \hline
        (7, 7) & 11.504372 & 43884 & (6, 2) & 2.071016 & 7393 \\
        \hline
        (4, 0) & 13.258387 & 52211 & (6, 3) & 1.585612 & 5695 \\
        \hline
        (1, 4) & 8.666449 & 35624 & (2, 0) & 0.994522 & 3671 \\
        \hline
        (7, 6) & 11.038945 & 47129 & (1, 1) & 1.226644 & 4705 \\
        \hline
        (4, 6) & 8.912769 & 40016 & (3, 7) & 1.217893 & 4968 \\
        \hline
        (2, 1) & 8.762945 & 39649 & (7, 5) & 1.064265 & 4432 \\
        \hline
        (4, 7) & 7.298406 & 34770 & (0, 4) & 0.331787 & 1397 \\
        \hline
        (0, 0) & 4.024789 & 20193 & (1, 0) & 0.163105 & 719 \\
        \hline
        (7, 3) & 2.349871 & 11774 & (7, 2) & 0.178921 & 858 \\
        \hline
        (6, 1) & 1.40726 & 7741 & (0, 2) & 0.164144 & 814 \\
        \hline
        (0, 3) & 0.595515 & 3571 & (6, 0) & 0.131961 & 689 \\
        \hline
        (5, 0) & 0.540275 & 3338 & (3, 6) & 0.077873 & 442 \\
        \hline
        (2, 7) & 0.277857 & 1861 & (2, 6) & 0.061833 & 375 \\
        \hline
        (0, 5) & 0.116634 & 838 & (0, 6) & 0.019758 & 130 \\
        \hline
        (0, 1) & 0.040778 & 332 & (1, 6) & 0.007116 & 52 \\
        \hline
        (7, 0) & 0.011538 & 121 & (5, 7) & 0.00296 & 27 \\
        \hline
        (0, 7) & 0.001524 & 30 & (1, 7) & 0.000286 & 5 \\
        \hline
        (6, 7) & 0.000165 & 4 & (7, 1) & 6.4e-05 & 2 \\
        \hline
    \end{tabular}
\end{center}
\texttt{Zwycięstwo gracza 1. Tur: 60}

Programy odwiedzają najwięcej węzłów w środkowej fazie gry,
zaś na początku i końcu jest ich o wiele mniej.
Alfa-beta cięcie jest zdecydowanie szybszy - odwiedza o wiele mniej węzłów.
Nie były one sprawdzane, ponieważ i tak nie miałoby to sensu ze względu na brak poprawy wyniku.
Zmiana rozpoczynającego algorytmu nie ma wpływu na partię - algorytmy wykonują te same ruchy.
Ze względu na brak różnicy w "myśleniu" algorytmów a zdecydowanie szybszym czasem wykonania od teraz
wszystkie testy będą wykonywane Alfa-beta cięciem.

Porównanie zmierzających się ze sobą strategii: \\
\texttt{Głębokość: 3}
\begin{center}
    \begin{tabular}{| m{5em} | m{5em} | m{5em} |}
        \hline
        Gracz 1. & Gracz 2. & Zwycięstwo \\
        \hline
        controlling & aggressive & 2 \\
        \hline
        aggressive & balanced & 2 \\
        \hline
        controlling & balanced & 1 \\
        \hline
        aggressive & controlling & 2 \\
        \hline
        balanced & aggressive & 2 \\
        \hline
        balanced & controlling & 2 \\
        \hline
    \end{tabular}
\end{center}

W przypadku pojedynku balans a kontrola wygrywa zawsze kontrola.
Mecze z agresywnym zawodnikiem bywają chaotyczne - ciężko stwierdzić kto wygra.

Porównanie głębokości i wygranych:
\begin{center}
    \begin{tabular}{| m{5em} | m{5em} | m{5em} | m{5em} |}
        \hline
        Strategia & Gracz 1. & Gracz 2. & Zwycięstwo \\
        \hline
        controlling & 2 & 4 & 2 \\
        \hline
        aggressive & 2 & 4 & 2 \\
        \hline
        balanced & 2 & 4 & 2 \\
        \hline
        controlling & 4 & 2 & 1 \\
        \hline
        aggressive & 4 & 2 & 1 \\
        \hline
        balanced & 4 & 2 & 1 \\
        \hline
    \end{tabular}
\end{center}
Zdecydowanie wyższe prawdopodobieństwo na wygraną ma algorytm z większą głębokością.

\section{Podsumowanie}
Alfa-beta cięcie jest zdecydowanie szybszym algorytmem.
Nie traci on również na jakości rozwiązania.

Bardzo ważnym jest znalezienie dobrych metryk sytuacji stanu gry,
a następnie dobranie odpowiedniego wektoru wag. Jeśli to nie zwycięstwo jest celem a rekreacja,
to można stworzyć różne, ciekawe strategie przeciwko którym można staczać ciekawe rozgrywki.
Oprócz tego można wprowadzić "anty-strategie", dzięki którym można obniżyć poziom kompetencji
algorytmu i sprawić, że rozgrywka będzie mniej wymagająca dla nowych graczy.


\printbibliography
\end{document}