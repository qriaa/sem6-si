\documentclass[12pt, a4paper]{article}
\usepackage[polish]{babel}
\usepackage[T1]{fontenc}
\usepackage{amsmath}
\usepackage{amsfonts}

\title{Sztuczna inteligencja i inżynieria wiedzy \\ Lista 2}
\author{Gabriel Urbaniak 260428}
\date{\today}

\newtheorem{definition}{Definicja}

\begin{document}
\maketitle

\section{Problem}

\subsection{Reversi}
Reversi to gra:
\begin{itemize}
    \item dwuosobowa,
    \item o sumie zerowej,
    \item niekooperacyjna,
    \item całkowicie deterministyczna,
    \item w której każdy gracz ma pełną informację o stanie gry,
    \item o skończonym czasie rozgrywki.
\end{itemize}

Zadaniem gracza w Reversi jest zwycięstwo przez posiadanie najwiekszej ilości pionków na koniec gry.

Formalny opis gry Reversi:
\begin{definition}[Gracz]
    Gracz $\boldsymbol{G} \in \{B, C\}$ to gracz grający w grę Reversi.
    Może wykonać ruch po ruchu przeciwnika lub jako pierwszy, gdy $\boldsymbol{G} = B$
\end{definition}

\begin{definition}[Plansza]
    Plansza to macierz $\boldsymbol{P} = (p_{ij})_{1 \le i \le 8,\text{ } 1 \le j \le 8}$, gdzie $p$ to pole na tej planszy.
    Pole może mieć trzy różne stany: $ \boldsymbol{S} = \{0, B, C\}, p_{ij} \in \boldsymbol{S} $.
\end{definition}

\begin{definition}[Ruch]
    Gracz może się ruszyć przez postawienie swojego \\ pionka na dowolne z wolnych pól $p_{ij} = 0$
    zmieniając jego symbol na $B$ lub $C$ zależnie od symbolu gracza dopóki jest w stanie przejąć pionki przeciwnika.
    Jeśli nie jest w stanie przejąć ani jednym ruchem pionka przeciwnika, to ruch gracza jest pomijany.
\end{definition}

\begin{definition}[Przejęcie]
    Przy postawieniu pionka jednego z graczy $\boldsymbol{G}$ na pole $p_{ij}$ należy sprawdzić, czy sąsiadujące pola
    $p_{i \pm 1 j \pm 1}$ należą do przeciwnika. Jeśli tak, to wszystkie pionki przeciwnika w linii prostej w
    pionie, poziomie lub przekątnej macierzy przecinającą pole $p_{ij}$ w kierunku tego pionka przeciwnika zostają przejęte
    przez gracza $\boldsymbol{G}$, czyli zmienione na jego symbol, pod warunkiem, że linia ta kończy się również jego pionkiem.
\end{definition}

\begin{definition}[Koniec gry]
    Gra się kończy, gdy żaden z graczy $\boldsymbol{G}$ nie może wykonać ruchu.
    Zliczane są wszystkie pola planszy $\boldsymbol{P}$ według ich stanu $\boldsymbol{S}$.
\end{definition}

\subsection{Metoda}

\begin{definition}[Stan gry]
    Stan gry to $\boldsymbol{R_{s}} = (\boldsymbol{P}, \boldsymbol{G})$, gdzie $\boldsymbol{P}$ to plansza, a $\boldsymbol{G}$ to gracz,
    który miałby wykonywać teraz ruch. 
\end{definition}

\begin{definition}[Drzewo decyzyjne]
    Drzewo decyzyjne to $\boldsymbol{N} = (C, S, \boldsymbol{R_s}), \\
    C = \{\boldsymbol{N}_1, \boldsymbol{N}_2, ...,  \boldsymbol{N}_n\} \lor \emptyset,
    S \in \mathbb{R} $, gdzie N jest węzłem, a S wynikiem funkcji oceniającej dla danego węzła.
\end{definition}

\section{Podsumowanie}

\end{document}